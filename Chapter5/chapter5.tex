%!TEX root = ../thesis.tex
%\input{commands.tex}
%*******************************************************************************
%****************************** Third Chapter **********************************
%*******************************************************************************
\chapter{Conclusions}

% **************************** Define Graphics Path **************************
\ifpdf
    \graphicspath{{Chapter5/Figs/Raster/}{Chapter5/Figs/PDF/}{Chapter5/Figs/}}
\else
    \graphicspath{{Chapter5/Figs/Vector/}{Chapter5/Figs/}}
\fi


\par{
Genetic variation along with natural selection, has driven the evolutionary history on earth creating us and all other life we know. Much work has been done to assess the population variation across humans and other species and use that to link genotypes with phenotypes and infer evolutionary histories. Less work has used these as markers to disambiguate data in different problems in genomics.
}
\par{
ScRNAseq suffers from several error modes that arise from natural limitations of detections of small materials as well as from the strategy used to partition cells. Because cells contain miniscule amounts of mRNA, amplification methods must be used which incur technical artifacts between cells within an experiment and more across experiments. Because cells are loaded into droplets in a poisson process, in order to capture many singletons, the cell suspension must be in a concentration that will also randomly load two or more cells into a single droplet. And if cells lyse or if there is RNA in solution prior to partitioning, some reads will have cell barcodes of cells from which they did not originate. One experimental design promises to address each of these issues at once. Mixing cells from multiple individuals reduces the batch effects when comparing them, cross sample multiplets should be easier to identify and remove, and the skew in allele fractions away from those expected from a diploid genome may be used to measure the ambient RNA. In chapter 2, I presented souporcell, a computational tool which uses the genetic variation between individuals to cluster cells in a single cell RNAseq mixture of individuals by the variants expressed in the reads. Souporcell also calls doublets using the alleles in cells versus the alleles in each cluster. And souporcell estimates the amount of ambient RNA in the system by how far the allele fractions in the clusters vary from those one expects from a diploid organism. I validated and compared souporcell to the other relevant tools and found that it compared favorably against all of them including the previous gold standard, demuxlet, which requires more information \textit{a priori}. I believe this is due to the rigid model based system used in demuxlet versus the simple cluster center method which is robust to any unmodeled factors. Souporcell has already been used in several million+ cell experiments and has been externally validated using cell hashing. I believe that mixture experimental designs will only get more popular over time due to the advantages this strategy has. 
}

\par{
Long read sequencing has undoubtedly revolutionized genome assembly and has motivated large projects such as the Darwin Tree of Life and the Earth Biogenome project which seek to sequence and assemble high quality genomes for all multicellular eukaryotic organisms. This will provide the next generation of science with an invaluable resource, allow for insights into evolution not possible before, and serve as a conservation of biological information in an era of extinction unprecedented during humans' time on earth. While much progress has been made through both data improvements and algorithmic methods, some issues remain especially for small organisms and highly heterozygous genomes. In chapter 3, we present the first high quality assembly of a single mosquito. This was made possible by recent advances in library preparation for long read sequencing reducing the DNA requirements. This improved the assembly versus other mosquito assemblies which used pools of individuals or short reads because of the presence of only two haplotypes versus many haplotypes.
}

\par{
While many consider heterozygosity a hinderance to genome assembly, and most assemblers perform worse on highly heterozygous genomes, it can have some benefits as well. In chapter 4, I turn this idea that heterozygosity is bad on its head and instead use heterozygosity as an advantage by using the phasing consistency of reads across multiple heterozygous sites as a signal for physical linkage in both assembly and scaffolding. In doing this, I also describe two phasing algorithms (three, if phased assembly is counted). One of these algorithms is novel and has several benefits of general interest. It is both robust to, and can correct incorrect genotypes because it does not make the assumption that every input variant is heterozygous and has relaxed the discrete constraint shared by most phasing algorithms. Another benefit of this algorithm is that it scales well for polyploid because we use treat haplotypes as cluster centers and can trivially increase the number of clusters as the ploidy increases. We demonstrate our phased assembly and scaffolding on the lepidoptera \textit{Vanessa atalanta}. While we do not meet the contiguity of some modern assemblers and do not assemble sex chromosomes, these can be assembled separately. We believe that phased assembly is the correct solution to diploid and polyploid assembly.
}