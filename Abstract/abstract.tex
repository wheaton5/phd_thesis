% ************************** Thesis Abstract *****************************
% Use `abstract' as an option in the document class to print only the titlepage and the abstract.
\begin{abstract}

Genetic variation and natural selection have driven the evolutionary history on this planet and are responsible for creating us and all other life as we know it. Over the past several decades, the genomic revolution has allowed us to assess population variation across humans and other species and use that to link genotypes with phenotypes and infer evolutionary histories. In this thesis, I explore computational methods for using genetic variation to demultiplex and disambiguate complex data. 

In single cell RNAseq, problems of batch effects, doublets, and ambient RNA are each sources of noise that impede our ability to infer the functional states of cells and compare them between experiments. One new popular new experimental design promising to solve each of these while also reducing experimental costs is mixturing multiple individuals' cells into a single experiment. In chapter 2, I present a method for clustering cells by genotype, calling doublets, and using the cross-genotype signal in singletons to estimate and remove ambient RNA. I compare this methods to other existing methods including one that requires \textit{a priori} information about the genotypes, and two which do not. I find that my method outperforms each of these methods across a wide range of data parameters and sample types.

In genome assembly, the recent higher throughput and lower cost of long read sequencing has revolutionized our ability to create reference quality genomes and has revitalized the assembly community. Now, massive efforts are taking place in the Darwin Tree of Life project and the Earth Biogenome project to create reference genomes for all multicelular eukaryotic life. This will create a scientific resource for the next generation of biological science, will serve as a conservation of data that could otherwise be lost in this time of mass extinction, and will allow for a much more broad understanding of evolution and the evolutionary history of life on Earth. While much progress has been made in data quality and assembly algorithms, some problems still exist. Until recently, the DNA input requirements for long read sequencing technologies made it impossible to sequence single individuals of these species with long reads. Also, high heterozygosity makes assembly more difficult due to the inherent ambiguity between heterozygous sequence versus paralogous sequence when confronted with inexact homology. One solution to the DNA input requirements would be to pool individuals, but this only increases the heterozygosity of the sample and reduces assembly quality. In chapter 3, we present the first high quality assembly of a single mosquito using new library preparation methods with reduced DNA requirements. This reduces the number of haplotypes to two, improving the assembly quality. In chapter 4, we further address the problems brought on by heterozygosity in assembly. I present a suite of tools that use the phasing consistency of multiple heterozygous sequences as a signal for physical linkage, thus using genetic variation to our advantage rather than as a challenge to overcome. This tool creates phased, linked assemblies and phasing aware scaffolding. Further, I provide a tool for phasing aware scaffolding on existing assemblies. This includes a novel haplotype phasing algorithm with some unique beneficial properties. It is robust to non-heterozygous variants as input and can detect and correct those genotypes. And it naturally extends to polyploid genomes. 

\end{abstract}
