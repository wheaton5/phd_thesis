% ************************** Thesis Acknowledgements **************************

\begin{acknowledgements}      


It is often by luck or some other random vicissitudes of life through which the most opportunity and learning arise. In this, I would like to thank someone whose name I do not know for taking a semester off from Brown and opening up a single dorm room in technology house my sophomore year. And I would like to thank Jimmy Kaplowitz and Mike Katzourin for making that connection without which I would be a very different person today. 

There I found my first unofficial mentors including Sean Smith, Lincoln Quirk, and Lucia Ballard from whom I learned intensely through both work and play. Through pair programming sessions with these three, I learned more in hours than months on my own. 

The Brown Computer Science TA program was where I learned to teach and lead and where I came to understand that a topic you cannot teach is a topic you do not, yourself, understand. So I would first like to thank the founder of this program, Andy van Dam, who has been the driving force of not only this program, but the entire culture of the Brown Computer Science department for decades. Andy is an intense guy, but he also has a flair for the absurd. The undergraduate TA program brings a sense of ownership, membership, and community to the students who contribute to it. There I met my first official advisor, Sorin Istrail who saw much more potential in me than I saw in myself. I also met another mentor, Franco Preparata who I went on to work with for years to come. His creativity and infectious excitement for the work we did was perhaps what made me decide that science and computational biology was for me. I also met more friends and my first mentee who later really became another mentor for me. Dan Heller, or just "Heller" first came into my life as a student then as an applicant for a TA position under me in which he stated that "CS4 changed my life". At the time this seemed absurd, but in retrospect, it was absolutely true, and accepting his application changed my life as well. Heller's work ethic combined with his rare ability to combine practicality with rigor is an inspiration to me to this day. 

I joined a company called Nabsys, which, despite not succeeding in its goal, succeeded in bringing together a number of talented people from whom I continued my intellectual journey. Peter Goldstein taught me statistics as I now understand it. Rather, he taught me some of what I know and the rest we learned together by painstakingly unravelling some of the more esoteric papers (and theses) regarding our topic including several from Michael Waterman. At Nabsys I also met the most talented Biochemist I know, Brendan Galvin, who remains one of my closest friends and mentors. Brendan thinks of biochemical assays in a similar way to how I think of constructing an algorithm. Brendan is something of a stealth super contributor to the genomics and transcriptomics world. He is not particularly well known, but the field would be dramatically worse off without him. We went on to work together at another company and hopefully some day we will have the opportunity to work together again someday. Because of our distinct expertise, each of our understanding the possibilities and limitations of each other's fields, and our friendship, our collaborations have been some of the most productive of my life. 

Through another one of life's serendipitous moments, I took what I thought was a throw away interview at a stealth genomics company. I was late for the interview due to poor planning, but Michael Schnall-Levin didn't balk at this and picked me up and brought me to the interview. After giving my job talk, I signed the non disclosure agreements and they told me what they were building. It wasn't until later that evening that the implications and possibilities started to sink in, and I started to become very excited. I took the job at 10x Genomics and still to this day I have never been in such a concentrated group of intellectual firepower. Patrick Marks is both one of the most talented computational biologists I have met and also the best manager I have had. The computational biology team as a whole is excellent because people follow truly talented and compassionate people like Pat. David Jaffe is one such person who became a friend and mentor to me. I miss his laugh, which is so absurd that you know he really means it. The rest of the company is also excellent and I would like to thank Alex Wong for running a great software team, Chris Hindson for never failing to deliver the secret sauce -- the gel beads and oil for the microfluidic system, and Serge Saxanov and Ben Hindson for leading such a great company. It was a pleasure and honor to work there with such amazing people and create products that are still changing the face of biology today. 

10x Genomics also contributed significantly to my opportunities going forward. Without the reputation of 10x Genomics as an innovative biotech company, my experience at 10x, and the papers and patents I was able to contribute to while working at 10x, I almost certainly would not have been considered for a PhD at some of the schools I was. I chose the Sanger Institute and Cambridge primarily because of Richard Durbin. His is one of the few textbooks that I have read cover to cover and I have followed his work throughout the years. The way he thinks comes through in his work and many of his papers have not only been great contributions to the field, but mind expanding to me personally. 

They say to never meet your heroes, and I have often felt the truth of this adage. However, Richard Durbin, or "The Durbinator" as I sometimes refer to him, consistently exceeds his already tremendous reputation. His algorithmic intuition is bar none. And while I would not be presumptuous enough to claim that we think alike, I would at least say that we have a similar algorithmic style. That is only part of what makes working with him incredible. He has the ability to see a global, decades-long plan for genomics and biology as well as the ability to work directly with the minute details of any of the diverse projects his group is working on.

Mara Lawniczak took me under her wing when I was struggling and gave me a home lab in which I could thrive. She has also been a true mentor to me in academia and life. I'd also like to thank her talented lab members including Ginny Howick, Arthur Talman, Juli Cudini, and others for their help and friendship. I'd also like to thank Sangjin Lee for his encouragement, support, and collaboration during the Covid19 pandemic. Without our pair programming sessions I would have been far less productive and less happy than I have been. During this past year, these sessions are often my only human contact in a given day. 

Obviously I owe my family everything. They instilled in me a love of science, culture, literature, and art and have supported me every step of the way even when they didn't agree with some of my decisions. Thanks especially to my mother for trying to make the pandemic as good as possible for me.

And finally I'd like to thank my cat, Kasparov, for being a very cute kitty.












\end{acknowledgements}
