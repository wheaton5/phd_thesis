% ************************** Thesis Acknowledgements **************************

\begin{acknowledgements}      


It is often by luck or some of the other random vicissitudes of life through which the most opportunity and learning arise. In this, I would like to thank someone whose name I do not know for taking a semester off from Brown and opening up a single dorm room in technology house my sophomore year. And I would like to thank Jimmy Kaplowitz and Mike Katzourin for making that connection without which I would be a very different person today. 

There I found my first unofficial mentors including Sean Smith, Lincoln Quirk, and Lucia Ballard from whom I learned intensely through both work and play. Through pair programming sessions with them, I learned more in hours than in months on my own. 

The Brown Computer Science teacher assistant program was where I learned to teach and lead. It is also where I learned that a topic you cannot coherently teach is a topic you do not, yourself, understand. So I would first like to thank the founder of this program, Andy van Dam, who has been the driving force of not only this program, but the entire culture of the Brown University Computer Science department since its inception. Andy is an intense guy, but he also has a flair for the absurd. The undergraduate TA program brings a sense of ownership, membership, and community to the students who contribute to it. There I met my first official advisor, Sorin Istrail, who saw much more potential in me than I saw in myself. I also met another mentor, Franco Preparata, who I went on to work with for years to come even after graduating. His creativity and infectious excitement for the work we did was perhaps what made me decide that science and computational biology was for me. I also met more friends and my first mentee who later really became another mentor for me. Dan Heller, or just ``Heller'', first came into my life as a student then as an applicant for a TA position under me in which he stated that ``CS4 changed my life''. At the time this seemed absurd, but in retrospect, it was absolutely true, and accepting his application changed my life as well. Heller's work ethic combined with his rare ability to balance practicality with rigor is an inspiration to me to this day. 

Another influence in my life was joining a company called Nabsys, which, despite not succeeding in its goal, succeeded in bringing together a number of talented people from whom I continued my intellectual journey. Peter Goldstein taught me much of statistics as I now understand it. At Nabsys I also met the most talented Biochemist I know, Brendan Galvin, who remains one of my closest friends and mentors. Brendan thinks of biochemical assays in a similar way to how I think of designing algorithms. Brendan is something of a stealth super contributor to the genomics and transcriptomics world. He is not particularly well known, but the field would be dramatically worse off without him. We went on to work together at 10x Genomics and hopefully some day we will have the opportunity to work together again. Our distinct yet semi-overlapping expertises allowed us to understand both the possibilities and limitations of each other's fields. This cross disciplinary understanding and communication was responsible for some of the most productive collaborations of my life thus far.

Through another one of life's serendipitous moments, I took what I thought was a throw away interview at a stealth genomics company. I decided to walk to the interview two miles away which turned out to be four miles away and so was late, but so was the hiring manager, Michael Schnall-Levin, who didn't balk at this and picked me up on his way in and brought me to the interview. After giving my job talk, I signed the non disclosure agreements and they told me about the technology they were building. It wasn't until later that evening that the implications and possibilities started to sink in, and I became very excited. I took the job at 10x Genomics and still to this day I have never been in such a concentrated group of intellectual firepower. Patrick Marks is both one of the most talented computational biologists I have met and also the best manager I have had. The computational biology team as a whole is excellent because people follow truly talented and compassionate people like Pat. David Jaffe is one such person who became a friend and mentor to me. I miss his laugh, which is so absurd and loud that you are assured of its authenticity. The rest of the company is also excellent and I would like to thank Alex Wong for running a great software team, Chris Hindson for never failing to deliver the secret sauce---the gel beads and oil for the microfluidic system---and Serge Saxonov and Ben Hindson for leading such a great company. It was a pleasure and honor to work with such amazing people and create products that are still changing the face of biology today. 

10x Genomics also contributed significantly to my opportunities going forward. Without the reputation of 10x Genomics as an innovative biotech company, and the papers and patents I was able to contribute to while working there, I almost certainly would not have been considered for a PhD at some of the schools I was. I chose the Sanger Institute and Cambridge primarily because of Richard Durbin. The way he thinks comes through in his work and many of his papers have not only been great contributions to the field, but mind expanding to me personally. 

There is a wise saying that you should never meet your heroes, and I have often felt the truth of this adage. However, Richard Durbin, or ``The Durbinator'' as I sometimes refer to him, consistently exceeds his already tremendous reputation. His algorithmic intuition is bar none. And while I would not be presumptuous enough to claim that we think alike, I would at least like to think that we have a similar algorithmic style. That is only part of what makes working with him incredible. He has the ability to see a global, decades-long plan for genomics and biology as well as the ability to work directly with the minute details of any of the diverse projects his group is working on.

Mara Lawniczak took me under her wing when I was struggling and gave me a home lab in which I could thrive. She has also been a true mentor to me in academia and life. I'd also like to thank her talented lab members including Ginny Howick, Arthur Talman, Juli Cudini, and others for their help and friendship. I'd also like to thank Sangjin Lee for his encouragement, support, and collaboration during the Covid19 pandemic. Without our pair programming sessions I would have been far less productive and less happy than I have been. During this past year, these sessions are often my only human contact in a given day. 

Obviously I owe my family everything. They instilled in me a love of science, culture, literature, the arts, and have supported me in all of my endeavors. Thanks especially to my mother who has always been my biggest fan, supporter, editor, interior designer, cooking (and eating) collaborator, and overall life advisor. I think she is both correct and completely unbiased in her opinions of me. I remember I was hiking in the Big Basin redwood forest in the summer of 2015 when I received a phone call from her. She said if I wanted to get a PhD, I should probably start planning. I said "I was just thinking of that myself." And fast forward to now, here I am. Also thanks to my father, who, at 74 years old, has continued treating patients in the hospital as a cardiologist through the pandemic. He is the single most dedicated and hardest working person I know and has always been an inspiration to me. Growing up, I constantly felt the impact of his work because, without exception, he invariably treated and potentially saved the life of a family member of the person I was interacting with. 

Finally I'd like to thank my cat, Kasparov, for being a very cute kitty. But also thanks to my mother for secretly getting me a cat during the pandemic.












\end{acknowledgements}
